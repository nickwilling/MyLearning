\documentclass[UTF8]{ctexart}

\usepackage{geometry}
% 交叉引用
\usepackage{hyperref}

\usepackage{graphicx}

\graphicspath{{../figures/}}%图片在当前目录下的figures目录
% comma: 用逗号分隔多个引用; square:使用方括号; super:引用是上角标形式

\geometry{a4paper, margin = 1in}

\bibliographystyle{plain}

\title{参考文献的使用}
\author{测试}
\date{\today}
\begin{document}
	    \maketitle
	    
	
		All fire behavior calculations assume that fuel moisture, wind speed,
		and wind direction are constant in time.\cite{finney_overview_2006}
		
		The model is inspired from GoogleNet architecture, considering its reasonable computational complexity and
		suitability for the intended problem compared to other computationally expensive networks such as AlexNet.
		To balance the effciency and accuracy, the model is fine-tuned considering the nature of the target problem
		and fire data.\cite{K.Muhammad_2018}
	    
		\section{研究背景模板}
		
		For instance,
		Chen et al. [6] investigated XXX for fire
		detection. Since, their method considers XXX, hence, it fails to XXX. 
		
		Besides XXX 方法/模型,谁谁谁 explored XXX方法 for prediction of XXX. 
		
		A similar method is proposed by 谁谁谁 by investigating temporal and spatial
		wavelet analysis, however, the excessive use of parameters by
		this method limits its usefulness.
		
		Another method is presented
		by 某某(Han and Lee [10]) by 通过什么途径【comparing the video frames and
		their color features for flame detection in tunnels.】 
	
		Considering these limitations, [Borges
		and Izquierdo [12]] attempted to 尝试做什么【detect fire】 using 使用什么方法【a multimodal
		framework consisting of color, skewness, and roughness features
		and Bayes classifier.】
		
		In continuation with 谁谁谁的工作 [Borges and Izquierdo [12] work],
		什么什么方法[multi-resolution 2D wavelets combined with energy and
		shape] are explored by Rafiee et al. [13]  为了尝试做什么【in an attempt to
		reduce false warnings】, however, 【存在什么不足】 the false fire alarms still
		remained significant due to movement of rigid body objects in
		the scene.
		
		An improved version of this approach is presented
		in [14] 【用什么方法代替什么方法 【using YUC instead of RGB color model】, 获得了更好的结果 providing
		better results than [13].
		
		什么方法在哪篇文章里被提出了【 Another color based flame detection
		method with speed 20 frames/sec is proposed in [15].】  这种结构使用了什么方法,在什么情况下获得了好效果( This
		scheme used SVM classifier to detect fire with good accuracy
		at smaller distance.】 在什么情况下效果比较差【  The method showed poor performance
		when fire is at larger distance or the amount of fire is comparatively
		small.	】
		
		\section{总结上面的传统方法的研究}
		还有进步的空间 【 Although, the method dominated
		state-of-the-art flame detection algorithms, yet there is still space for improvement.】
		
		存在什么缺陷 【In addition, the false alarming rate
		is still high and can be further reduced.】
		
		提出动机,开发这种方法是有需要的【 there is a need to develop fire detection
		algorithms with less computational cost and false warnings,
		and higher accuracy.】
		
		
	    根据以上的动机,开展了什么样的研究 【 Considering the above motivation, we
		extensively studied convolutional neural networks (CNNs)
		for flame detection at early stages in CCTV surveillance
		videos.】
		
		\section{提出论文的主要贡献}
		The main contributions of this article are summarized
		as follows:
		
		考虑到传统方法的局限性,【Considering the limitations of traditional handengineering
		methods,】研究了什么 【 we  extensively studied deep learning
		(DL) architectures for this problem】  提出了什么方法【 and propose a
		cost-effective CNN framework for flame detection in
		CCTV surveillance videos.】 这种方法相比传统方法有什么好处。【Our framework avoids the
		tedious and time consuming process of feature engineering
		and automatically learns rich features from raw fire
		data.】
		
		受什么牛逼的技术启发,干了什么事情【 Inspired from transfer learning strategies, we trained
		and fine-tuned a model with architecture similar to
		GoogleNet [18] for fire detection,】 有什么成果 【 which successfully
		dominated traditional fire detection schemes.】
		
		虽然比不过state-of-art,但是提出的方法在保证了准确率,和有比较低的计算复杂性的同时又在上述的Motivation的点里有比较好的性能【 The proposed framework balances the fire detection accuracy
		and computational complexity as well as reduces the
		number of false warnings compared to state-of-the-art fire
		detection schemes.】 所以提出的模型更适合干什么。【 Hence, our scheme is more suitable for
		early flame detection during surveillance to avoid huge
		fire disasters.】
		
		\section{文章的组织架构}
		The rest of the paper is organized as follows: In Section 2,
		we present our proposed architecture for early flame detection
		in surveillance videos. Experimental results and discussion
		are given in Section 3. Conclusion and future directions
		are given in Section 4.
		
		
		\section{数据集相关}
		The total
		number of images used in experiments is 68457, out of
		which【 其中 】 62690   frames are taken from Dataset1 and remaining【 剩下的 】
		from other sources.
		
		\section{图表写法}
		Figure 5 shows sample images from this dataset. 
		
		Table 1 shows the experimental results based on Dataset1 and its
		comparison with other methods.
		
		\begin{figure}
			\centering
			\includegraphics[width=0.7\linewidth]{comparison.png}
			\caption{Comparison with indifferent fire detection methods.}
			\label{fig:}
		\end{figure}
	
	\section{实验}
	In Figure 7 (b),
	the fire region in the image is 【 做了什么处理,比如加了噪音】 distorted and the 因此产生的图片【 resultant image】
	is passed through our method.
		
		
	\section{可能用到的句子}
	In conventional fire
	detection systems, sensors used to detect fires are heat
	detectors and smoke detectors that have an important role in
	fire detection process.\cite{Riyadi_2018}
	
	
	
	如果检测到了火焰就会发警报If the presence of flame is detected by system, the
	system wil produce an alarm as an indicator.
	
	现在的火灾检测还是依靠传统方法和直觉,还没有现成的基于数据驱动的方法。However, AFRD's fire inspection practices relied on tradition and intuition, with no exist-ing data-driven process for prioritizing fire inspections or
	identifying new properties requiring inspection. \cite{madaio_firebird:_2016}
	
	【 帮助其做出明智的决定 】 Firebird integrates and visualizes fire incidents, property information and risk scores to help AFRD make informed decisions about fire inspections.
	
	虽然城市火灾风险预测是很重要的,但是相比于其他风险预测,没有得到足够的关注。Risk prediction models have been widely used in many domains, including health care [15], student performance evaluation [16], and accounting fraud detection [18].However, urban fire risk prediction has received relatively less atten-
	tion, despite its obvious importance.
	
	大多数有关数据驱动的火灾风险预测都定位在了森林火灾预测。Much of the prior work on data-driven fire risk prediction has targeted woodland and forest fires, such as in Italy  [17], Greece [14], and Portugal [9]。

	The features they used, such as soil type and topography, are very different from the ones typically used in urban fire prediction like
	construction material and property usage type.
	
	17【 A  KDD based multicriteria decision making model for
	re risk evaluation. 】
	
	14 【 A decision support system applying an
	integrated fuzzy model for long-term forest re risk
	estimation.】 
	
	9【	Spatial prediction of fire ignition
	probabilities: comparing logistic regression and neural
	networks. 】
	
	
	现在在属性或建筑级别的火灾风险预测的相关研究还非常少。There is limited work on predicting fire risk at the property or building level.
	
	大多数火灾始于初期,并进一步发展到闷烧,燃烧和火灾阶段。Most fires start from an incipient stage and develop further to smouldering, flaming and fire stages.\cite{andrew_multi-stage_2016}
	
	在火灾的初期和闷烧阶段火焰和烟雾还比较少,等到燃烧和火灾阶段,火势更加猛烈,火焰和烟雾程度更深,并散发极高的热量。\cite{andrew_multi-stage_2016}
	
	火灾数据分析最常见的方法是聚类和分类算法。The most common methods used are related to clustering techniques and classification algorithms.
	
	检测初期火灾阶段的传感器需要被放置在离地面2.1米高处(天花板),因为实验结果表明\cite{andrew_multi-stage_2016},火灾初起产生的气体会首先充满整个房间的顶部,由于火灾产生的气体的密度要比周围空气轻。
	
	传感器采集频率:For realisation of a wireless sensing IAQ system, the data of the low cost system is sampled at the sampling rate of 10 sample/min [15]. The data has been recorded for 15 min each time。
	
	火灾的主要死伤原因就是有毒气体的吸入。\cite{molyneux2014correlation}UK fire statistics [7] show that the main cause of death, and the main cause of injury, in fires arises from the inhalation of toxic effluents.【Fire Statistics Great Britain 2011 to 2012 - Data on incidents attended by fire and rescue services across Great Britain, Department for Communities and Local Government, UK.】
	
	CO对人体产生的影响使用以(\%COHb)表示的吸入剂量来计算。在达到20\%COHb的时候就会对人体产生影响;到达30-40\%COHb 的时候会使人失去意识;而超过50\%COHb时就会致死。如果以吸入浓度来计算的话,最小的致人失去意识的CO浓度是暴露在 200 ppm的CO浓度下6小时(30\% COHb),然而在火灾中,CO的浓度通常会在1000-10000ppm,甚至更高,在几分钟内就会使人丧失行动能力。\cite{molyneux2014correlation}  The effects of CO depend upon the dose inhaled in terms of \%COHb. Effects are
	minimal up to approximately 20\%COHb, with incapacitation (loss of consciousness) occurring at 30-
	40\%COHb and death above ~50\%COHb. The minimum inhaled concentration capable of causing loss of
	consciousness is approximately 200 ppm after around 6 hours exposure (at 30\% COHb). Exposure to
	concentrations of 1 000-10 000 ppm CO is common in fires, and may cause incapacitation within a few
	minutes [10].
	
	ISO 13571[6] considers the four major hazards from fire
	which may prevent escape (toxic gases, irritant gases, heat and smoke obscuration). 【 ISO 13571:2012 Life-threatening components of fire – Guidelines for the estimation of time
	available for escape using fire data.】
	\bibliography{SmartFire}
\end{document}
